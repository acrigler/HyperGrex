%\documentclass{pbml}
\documentclass[nofonts]{pbml} % use default fonts
%\documentclass[color]{pbml} % for color images and hypertext links

\usepackage{graphicx}
\usepackage{multicol}
\usepackage{amssymb}
\usepackage{natbib}
\usepackage{euler}
\usepackage{latexsym}
\usepackage{verbatim}
\usepackage{amsmath}
\usepackage{tikz}
\usepackage{wrapfig}
\usepackage{colortbl}
\usepackage{xcolor}
\usetikzlibrary{fit,positioning}

\newcommand{\ensuretext}[1]{#1}
\newcommand{\mycomment}[3]{\ensuretext{\textcolor{#3}{[#1 #2]}}}
\newcommand{\ammarker}{\ensuretext{\textcolor{blue}{\ensuremath{^{\textsc{A}}_{\textsc{M}}}}}}
\newcommand{\am}[1]{\mycomment{\ammarker}{#1}{blue}}
\newcommand{\cjmarker}{\ensuretext{\textcolor{red}{\ensuremath{^{\textsc{C}}_{\textsc{D}}}}}}
\newcommand{\cjd}[1]{\mycomment{\cjmarker}{#1}{red}}
\newcommand{\ignore}[1]{}
\newcolumntype{C}{>{\centering\arraybackslash}p{0.6ex}}


\begin{document}
\title{Tree Transduction Tools for cdec}

\institute{label1}{Carnegie Mellon University}
\institute{label2}{University of Oxford}

\author{
  firstname=Austin,
  surname=Matthews,
  institute=label1,
}
\author{
  firstname=Paul,
  surname=Baltescu,
  institute=label2,
}
\author{
  firstname=Phil,
  surname=Blunsom,
  institute=label2,
}
\author{
  firstname=Alon,
  surname=Lavie,
  institute=label1,
}
\author{
  firstname=Chris,
  surname=Dyer,
  institute=label1,
  corresponding=yes,
  email={cdyer@cs.cmu.edu},
  address={Language Technologies Institute\\Carnegie Mellon University\\Pittsburgh, PA 15213, United States}
}

\PBMLmaketitle

\begin{abstract}
We describe a collection of open source tools for learning
tree-to-string and tree-to-tree transducers and the extensions to the
cdec decoder that enable translation with these. Our modular,
easy-to-extned tools extract rules from trees or forests aligned to
strings and trees subject to different structural constraints. A fast,
multithreaded implementation of the Cohn and Blunsom (2009) model for
extracting compact tree-to-string rules is also included. The
implementation of the tree composition algorithm used by cdec is
described, and translation quality and decoding time results are
presented. Our results add to the body of evidence suggesting that
tree transduers are a compelling option for translation, particularly
when decoding speed and translation model size are important.
\end{abstract}

\section{Introduction}

\section*{Acknowledgements}
This research is supported by \ldots

\bibliography{mtm2014}

\correspondingaddress
\end{document}
